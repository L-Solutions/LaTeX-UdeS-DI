
    Description du chapitre.

\section{L'informatique de gestion}

     L'informatique peut \^etre utilis\'ee pour r\'epondre \`a des besoins divers, tels que le contr\^ole de processus industriels, le calcul scientifique et technique, la gestion des organisations. Parmi ces applications, il en est une qui rev\^et une importance particuli\`ere : l'informatique appliqu\'ee \`a la gestion ou informatique de gestion. Il suffit pour s'en convaincre de constater que plus des trois quarts des calculateurs \'electroniques install\'es dans les pays d\'evelopp\'es sont employ\'es \`a r\'esoudre des probl\`emes de gestion. Ils constituent, \`a l'heure actuelle, une source de productivit\'e et un catalyseur d'\'evolution d'une port\'ee consid\'erable.

    Comme dans toute application de l'informatique, on fait ici intervenir un ensemble de mat\'eriels ayant chacun une fonction pr\'ecise ou hardware et des instructions rassembl\'ees en programmes constituant le software. La nature des travaux de gestion r\'ealis\'es par un syst\`eme d\'etermine dans une large mesure celle des mat\'eriels qui le constituent et des programmes qui assurent son fonctionnement. Bien que la sp\'ecificit\'e des traitements de l'information de gestion ne soit que relative, ce texte laissera de c\^ot\'e la technologie des syst\`emes informatiques ou les connaissances g\'en\'erales relatives au traitement de l'information par les machines \'electroniques, pour se limiter aux applications de celles-ci \`a la gestion et aux \'eventuelles cons\'equences de ces applications.

{\it Auteur: Ren\'e-Charles CROS}

\section{Titre de section}
        
    Les applications de l'informatique qui s'imposent \`a l'\'evidence concernent l'automatisation des travaux administratifs et comptables. Ce sont l\`a des op\'erations simples, p\'eriodiques, qui ne font pas intervenir de d\'ecisions et o\`u l'ordinateur remplace avantageusement l'homme. Dans une entreprise industrielle, la comptabilit\'e g\'en\'erale et analytique ainsi que les comptes clients et les comptes fournisseurs, l'\'etat des stocks, la facturation et la paie du personnel sont pris en charge par l'ordinateur. Il en est de m\^eme pour de nombreuses op\'erations comptables et administratives dans une banque, dans une compagnie d'assurances ou dans une administration telle que la Direction g\'en\'erale des imp\^ots.

    Le syst\`eme informatique peut \'egalement assurer des fonctions plus complexes. Ainsi, dans une entreprise industrielle, les commandes des clients peuvent \^etre traduites en ordre de fabrication en assurant le meilleur emploi de l'appareil de production et en g\'erant les stocks de fa\c{c}on satisfaisante. Dans ce cas, l'ordinateur pr\'epare les d\'ecisions selon des r\`egles pr\'ealablement fix\'ees par les programmes. Cela est possible chaque fois que les informations sont bien d\'efinies et lorsque les choix possibles sont connus ainsi que leurs cons\'equences. Au moyen des programmes \'ecrits par l'utilisateur, l'ordinateur, \`a mesure qu'il re\c{c}oit les informations, calcule les r\'esultats qui permettent de prendre les d\'ecisions ou qui d\'efinissent celles-ci.

    Dans d'autres cas, o\`u les d\'ecisions de gestion ne sont pas analysables avec une telle rigueur (comme, par exemple : \`a qui confier telle mission ou quel budget affecter \`a la publicit\'e ?), le syst\`eme informatique sera un outil d'information et d'analyse, puisque le responsable pourra ais\'ement consulter les divers fichiers et, \'eventuellement, testera telle d\'ecision en utilisant un mod\`ele.

    Les soci\'et\'es d\'evelopp\'ees sont caract\'eris\'ees en particulier par une circulation sans cesse croissante d'informations de toute nature ; aussi l'\'economie globale de ces flux d'informations est recherch\'ee afin d'assurer leur plein emploi. Cela entra\^ine le d\'eveloppement de syst\`emes informatiques collectifs et coop\'eratifs tr\`es divers. Ainsi, un syst\`eme permettant la r\'eservation de moyens de transports, de chambres d'h\^otel et de services analogues \`a partir d'un terminal unique constitue un outil dont les avantages sont multiples : meilleur emploi des ressources, co\^uts administratifs r\'eduits, commodit\'e pour l'usager... Dans un autre domaine, la simplification des \'echanges d'informations entre les personnes physiques et morales et les administrations (S\'ecurit\'e sociale, Finances, P.T.T., etc) sera assur\'ee par la cr\'eation d'une centrale automatis\'ee d'information assurant le recueil de celle-ci et sa communication \`a chaque demandeur sous la r\'eserve d'un contr\^ole et d'une r\'eglementation assurant un usage conforme aux droits et libert\'es de chacun.

    Ces aper\c{c}us mettent en \'evidence que, pratiquement, toute activit\'e, qu'elle soit \'economique, sociale ou intellectuelle, peut donner lieu \`a des applications de l'ordinateur. Celui-ci se verra confier, outre les t\^aches purement administratives, les cas o\`u les d\'ecisions sont pr\'evisibles et analysables. D'autre part, c'est un moyen d'information consid\'erable puisqu'il peut apporter \`a chacun la documentation souhait\'ee au moment opportun et sans attente. Enfin, comme prolongement de la m\'emoire et de la logique, il constitue pour l'homme une aide dans ses activit\'es les plus cr\'eatrices, qu'elles soient enseignement, recherche... Aujourd'hui, on peut affirmer que les m\'ethodes de travail de chacun seront modifi\'ees du fait de l'existence de ce nouvel outil ; selon les cas, leurs \'evolutions seront profondes ou secondaires, rapides ou progressives, mais elles seront partout pr\'esentes.
        
    \subsection{Essai d'une figure}
    
    \begin{figure}[h]
    Comme dans toute application de l'informatique, on fait ici intervenir un ensemble de mat\'eriels ayant chacun une fonction pr\'ecise ou hardware et des instructions rassembl\'ees en programmes constituant le software. La nature des travaux de gestion r\'ealis\'es par un syst\`eme d\'etermine dans une large mesure celle des mat\'eriels qui le constituent et des programmes qui assurent son fonctionnement. Bien que la sp\'ecificit\'e des traitements de l'information de gestion ne soit que relative, ce texte laissera de c\^ot\'e la technologie des syst\`emes informatiques ou les connaissances g\'en\'erales relatives au traitement de l'information par les machines \'electroniques, pour se limiter aux applications de celles-ci \`a la gestion et aux \'eventuelles cons\'equences de ces applications.
		\caption{Un morceau de texte en figure.}
    \end{figure}
    
    \subsection{Titre de sous-section}

    Sous-section.
        
\section{Titre de section}

    Section.
        
    \subsection{Titre de sous-section}
    
    Sous-section.
    
    \subsection{Titre de sous-section}
    
    Sous-section.
    
    \subsubsection{Titre de sous-sous-section}
    
    Sous-sous-section.
    
    \paragraph{Un paragraphe:} exemple
    
    \subparagraph{Un sous paragraphe:} exemple
    
    \subsubsection{Titre de sous-sous-section}

    Sous-sous-section.
    
    \subsection{Titre de sous-section}
    
    Sous-section.
