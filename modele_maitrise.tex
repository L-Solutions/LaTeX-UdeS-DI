%!TEX TS-program = pdflatex

\documentclass[caractereUtf, % caract\`ere UTF8 activ\'e
               maitrise,     % mod\`ele pour une ma\^itrise
               hypertexte,   % support pour les liens 
               simpleface    % impression sur simple face
               ]{style/scienceUdeS}

%% Gabarit pour maitrise et pour maitrise GL
%%
%% Version 2019/03/25 v2.1
%%
%% Benoît Fraikin
%% benoit.fraikin(at)usherbrooke(dot)ca

% =========================================== Options principales
% - bibliothequeNationale : pr\'esente le document pour la copie de la 
%   biblioth\`eque nationale. Attention, certaines commandes ne
%   fonctionnent plus.
%
% - Natbib : d\'eclare le package Natbib pour \^etre utilis\'e dans le m\'emoire
%
% - caractereUtf : active les caract\`eres cod\'es en UTF8 au lieu du latin1
% - caractereLatin : active les caract\`eres cod\'es en latin1 au lieu du UTF8
%
% - logopdflatex : utilise le logo de l'universit\'e en pdf, pour ceux 
%   qui utilise pdflatex et non latex.
%
% - pasDeFigure : le document ne contient pas de figure (\begin{figure})
% - pasDeTableau : le document ne contient pas de tableau (\begin{table})
% - pasDeListing : le document ne contient pas de listing (\begin{lstlisting})
%
% - maitrise : pour une ma\^itrise en informatique
% - maitriseGL : pour une ma\^itrise en g\'enie logiciel
%                (ne diff\`ere pas de l'option "maitrise" pour l'instant)
% - these : pour une these de doctorat
%
% - enRedaction : option pour signaler l'\'etat du document
%                 \`a retirer pour la version finale
% =========================================== Options secondaires
% - hypertexte
% - rectoverso
% - simpleface
% - draft
% - final
%=================================================================

\begin{document}
%====================== DEBUT DU DOCUMENT ========================
\modeFrancais
% -------------------------------------------------
% PACKAGE SUPPLEMENTAIRE 
% -------------------------------------------------

% Les packages d\'ej\`a charg\'es sont les suivants :
%--> \usepackage[utf8]{inputenc} ou \usepackage[latin1]{inputenc}
%--> \usepackage[T1]{fontenc}
%--> \usepackage{lmodern}
%--> \usepackage{enumerate}
%--> \usepackage{amsfonts, amsmath, amssymb, stmaryrd, latexsym}
%--> \usepackage{xspace, setspace}
%--> \usepackage{array}
%--> \usepackage[dvipsnames,usenames]{color}
%--> \usepackage[table]{xcolor}
%--> \usepackage{wrapfig, epsfig}
%--> \usepackage[english,frenchb]{babel}
%--> \usepackage{frbib} % bibliography en francais
%--> \usepackage{fancyhdr}
%--> \usepackage{geometry}
%--> \usepackage{ulem} % for sout and underline
%--> \renewcommand{\emph}{\textit}
%--> \usepackage{times,helvet,courier}
%--> \usepackage{listings}

% Vous pouvez ajouter des nouveaux packages ici
% \

% -------------------------------------------------
% COMMANDE SUPPLEMENTAIRE
% -------------------------------------------------

%---- pour l'anglais ----
\newcommand{\ie}{{i.e.}\xspace}
\newcommand{\eg}{{e.g.}\xspace}
\newcommand{\cf}{{\it cf.}\xspace}

%---- pour le francais ----
\newcommand{\cad}{{c'est-\`a-dire}\xspace}
\newcommand{\Cad}{{C'est-\`a-dire}\xspace}
\newcommand{\etc}{{\it etc.}\xspace}

%---- Math ----
\newcommand{\N}{\ensuremath{\mathbb{N}}\xspace}
\newcommand{\Z}{\ensuremath{\mathbb{Z}}\xspace}

% Nouvelles d\'efintion d'environnement de th\'eor\`eme et de d\'efinition.
\newtheorem{frtheoreme}{Th\'eor\`eme}[section]
\newtheorem{frlemme}[frtheoreme]{Lemme}
\newtheorem{frprop}[frtheoreme]{Proposition}
\newtheorem{frcoro}[frtheoreme]{Corollaire}
\newtheorem{frdefinition}{D\'efinition}[section]

% -------------------------------------------------
% PAGE DE TITRE
% -------------------------------------------------

% ATTENTION : ne pas utiliser les macros \title \author ou \date !!!
\auteur{Beno\^it Fraikin}
\titre{Ma ma\^itrise ou ma th\`ese}

% -------------------------------------------------

% DESCRIPTION DU SOMMAIRE (EN FRANCAIS) -----------
\MotsCles{mots-cles1; mots-cles2.}
\sommaire{Ceci est le sommaire. Il est en fran\c{c}ais.}

% REMERCIEMENTS -----------------------------------
\remerciements{Ceci est la liste des remerciements. Ils sont en fran\c{c}ais.}

% LISTE DES ABREVIATIONS --------------------------
\abreviations{ 
    \begin{description}
    \item[SW] Star Wars
    \item[ST] Star Trek
    \item[BV] Babylon V
    \end{description}
}

% !TEX root =  ../modele_these.tex
% DEBUT DE LA PAGE DU JURY

\DecisionDuJury
{%
\today
}%
{%
\Directeur{[Pr\'enom et Nom]}{[Nom]}
%\Directrice{[Pr\'enom et Nom]}{[Nom]}

\Codirectrice{[Pr\'enom et Nom]}{[Nom]}
%\Codirecteur{[Pr\'enom et Nom]}{[Nom]}

\MembreInterne{[Pr\'enom et Nom]}{[Nom]}

\MembreExterne{[Pr\'enom et Nom]}{[Titre]}{[Nom]}{[provenance]}

\President{[Pr\'enom et Nom]}{[Nom]}
}%
% FIN DE LA PAGE DU JURY


%=================================================================
\modeFrancais{}
\enteteDeLaThese{}
\renewcommand{\chaptermark}[1]{\markboth{\textsc{\chaptername\ \thechapter.\ #1}}{}}
\renewcommand{\sectionmark}[1]{\markright{\textsc{\thesection. #1}}	}
    
%========================== INTRODUCTION ===========================

\modeFrancais{}
% DEBUT DE L'INTRODUCTION
\Introduction
    Ceci est l'introduction.
    Ceci est l'introduction.
    Ceci est l'introduction.
    Ceci est l'introduction.
    Ceci est l'introduction.
    Ceci est l'introduction.
    Ceci est l'introduction.
    Ceci est l'introduction.
    Ceci est l'introduction.
    Ceci est l'introduction.
    Ceci est l'introduction.
    Ceci est l'introduction.
    Ceci est l'introduction.
    Ceci est l'introduction.
    Ceci est l'introduction.
    Ceci est l'introduction.
    Ceci est l'introduction.
    Ceci est l'introduction.
    Ceci est l'introduction.
    Ceci est l'introduction.
    Ceci est l'introduction.
    Ceci est l'introduction.
    Ceci est l'introduction.
    Ceci est l'introduction.
    Ceci est l'introduction.
    Ceci est l'introduction.
    Ceci est l'introduction.
    Ceci est l'introduction.
    Ceci est l'introduction.
    Ceci est l'introduction.
    Ceci est l'introduction.
    Ceci est l'introduction.
    Ceci est l'introduction.
    Ceci est l'introduction.
    Ceci est l'introduction.
    Ceci est l'introduction.
    Ceci est l'introduction.
    Ceci est l'introduction.
    Ceci est l'introduction.
    Ceci est l'introduction.
    Ceci est l'introduction.
    Ceci est l'introduction.
    Ceci est l'introduction.
    Ceci est l'introduction.
    Ceci est l'introduction.
    Ceci est l'introduction.
    Ceci est l'introduction.
    Ceci est l'introduction.
    Ceci est l'introduction.
    Ceci est l'introduction.
    Ceci est l'introduction.
    Ceci est l'introduction.
    Ceci est l'introduction.
    Ceci est l'introduction.
    Ceci est l'introduction.
    Ceci est l'introduction.
    Ceci est l'introduction.
    Ceci est l'introduction.
    Ceci est l'introduction.
    Ceci est l'introduction.
    Ceci est l'introduction.
    Ceci est l'introduction.
    Ceci est l'introduction.
    Ceci est l'introduction.
    Ceci est l'introduction.
    Ceci est l'introduction.
    Ceci est l'introduction.
    Ceci est l'introduction.
    Ceci est l'introduction.
    Ceci est l'introduction.
    Ceci est l'introduction.
    Ceci est l'introduction.
    Ceci est l'introduction.
    Ceci est l'introduction.
    Ceci est l'introduction.
    Ceci est l'introduction.
    Ceci est l'introduction.
    Ceci est l'introduction.
    Ceci est l'introduction.
    Ceci est l'introduction.
    Ceci est l'introduction.
    Ceci est l'introduction.
    Ceci est l'introduction.
    Ceci est l'introduction.
    Ceci est l'introduction.
    Ceci est l'introduction.
    Ceci est l'introduction.
    Ceci est l'introduction.
    Ceci est l'introduction.
    Ceci est l'introduction.
    Ceci est l'introduction.
    Ceci est l'introduction.
    Ceci est l'introduction.
    Ceci est l'introduction.
    Ceci est l'introduction.
    Ceci est l'introduction.
    Ceci est l'introduction.
    Ceci est l'introduction.
    Ceci est l'introduction.
    Ceci est l'introduction.
    Ceci est l'introduction.
    Ceci est l'introduction.
    Ceci est l'introduction.
    Ceci est l'introduction.
    Ceci est l'introduction.
    Ceci est l'introduction.
    Ceci est l'introduction.
    Ceci est l'introduction.
    Ceci est l'introduction.
    Ceci est l'introduction.
    Ceci est l'introduction.
    Ceci est l'introduction.
    Ceci est l'introduction.
    Ceci est l'introduction.
    Ceci est l'introduction.
    Ceci est l'introduction.
    Ceci est l'introduction.
    Ceci est l'introduction.
    Ceci est l'introduction.
    Ceci est l'introduction.
    Ceci est l'introduction.
    Ceci est l'introduction.
    Ceci est l'introduction.
    Ceci est l'introduction.
    Ceci est l'introduction.
    Ceci est l'introduction.
    Ceci est l'introduction.
    Ceci est l'introduction.
    Ceci est l'introduction.
    Ceci est l'introduction.
    Ceci est l'introduction.
    Ceci est l'introduction.
    Ceci est l'introduction.
    Ceci est l'introduction.
    Ceci est l'introduction.
    Ceci est l'introduction.
    Ceci est l'introduction.
    Ceci est l'introduction.
    Ceci est l'introduction.
    Ceci est l'introduction.
    Ceci est l'introduction.
    Ceci est l'introduction.
    Ceci est l'introduction.
    Ceci est l'introduction.
    Ceci est l'introduction.
    Ceci est l'introduction.
    Ceci est l'introduction.
    Ceci est l'introduction.
    Ceci est l'introduction.
    Ceci est l'introduction.
% FIN DE L'INTRODUCTION
   
% Un article par chapitre
%=========================== CHAPITRE 1 ============================

\modeFrancais{}
\chapter[Titre court du chapitre 1] 
        {\singlespacing%
         Titre long du chapitre 1}
         \label{ch:chapitre-1}

    Description du chapitre.

\section{L'informatique de gestion}

     L'informatique peut \^etre utilis\'ee pour r\'epondre \`a des besoins divers, tels que le contr\^ole de processus industriels, le calcul scientifique et technique, la gestion des organisations. Parmi ces applications, il en est une qui rev\^et une importance particuli\`ere : l'informatique appliqu\'ee \`a la gestion ou informatique de gestion. Il suffit pour s'en convaincre de constater que plus des trois quarts des calculateurs \'electroniques install\'es dans les pays d\'evelopp\'es sont employ\'es \`a r\'esoudre des probl\`emes de gestion. Ils constituent, \`a l'heure actuelle, une source de productivit\'e et un catalyseur d'\'evolution d'une port\'ee consid\'erable.

    Comme dans toute application de l'informatique, on fait ici intervenir un ensemble de mat\'eriels ayant chacun une fonction pr\'ecise ou hardware et des instructions rassembl\'ees en programmes constituant le software. La nature des travaux de gestion r\'ealis\'es par un syst\`eme d\'etermine dans une large mesure celle des mat\'eriels qui le constituent et des programmes qui assurent son fonctionnement. Bien que la sp\'ecificit\'e des traitements de l'information de gestion ne soit que relative, ce texte laissera de c\^ot\'e la technologie des syst\`emes informatiques ou les connaissances g\'en\'erales relatives au traitement de l'information par les machines \'electroniques, pour se limiter aux applications de celles-ci \`a la gestion et aux \'eventuelles cons\'equences de ces applications.

{\it Auteur: Ren\'e-Charles CROS}

\section{Titre de section}
        
    Les applications de l'informatique qui s'imposent \`a l'\'evidence concernent l'automatisation des travaux administratifs et comptables. Ce sont l\`a des op\'erations simples, p\'eriodiques, qui ne font pas intervenir de d\'ecisions et o\`u l'ordinateur remplace avantageusement l'homme. Dans une entreprise industrielle, la comptabilit\'e g\'en\'erale et analytique ainsi que les comptes clients et les comptes fournisseurs, l'\'etat des stocks, la facturation et la paie du personnel sont pris en charge par l'ordinateur. Il en est de m\^eme pour de nombreuses op\'erations comptables et administratives dans une banque, dans une compagnie d'assurances ou dans une administration telle que la Direction g\'en\'erale des imp\^ots.

    Le syst\`eme informatique peut \'egalement assurer des fonctions plus complexes. Ainsi, dans une entreprise industrielle, les commandes des clients peuvent \^etre traduites en ordre de fabrication en assurant le meilleur emploi de l'appareil de production et en g\'erant les stocks de fa\c{c}on satisfaisante. Dans ce cas, l'ordinateur pr\'epare les d\'ecisions selon des r\`egles pr\'ealablement fix\'ees par les programmes. Cela est possible chaque fois que les informations sont bien d\'efinies et lorsque les choix possibles sont connus ainsi que leurs cons\'equences. Au moyen des programmes \'ecrits par l'utilisateur, l'ordinateur, \`a mesure qu'il re\c{c}oit les informations, calcule les r\'esultats qui permettent de prendre les d\'ecisions ou qui d\'efinissent celles-ci.

    Dans d'autres cas, o\`u les d\'ecisions de gestion ne sont pas analysables avec une telle rigueur (comme, par exemple : \`a qui confier telle mission ou quel budget affecter \`a la publicit\'e ?), le syst\`eme informatique sera un outil d'information et d'analyse, puisque le responsable pourra ais\'ement consulter les divers fichiers et, \'eventuellement, testera telle d\'ecision en utilisant un mod\`ele.

    Les soci\'et\'es d\'evelopp\'ees sont caract\'eris\'ees en particulier par une circulation sans cesse croissante d'informations de toute nature ; aussi l'\'economie globale de ces flux d'informations est recherch\'ee afin d'assurer leur plein emploi. Cela entra\^ine le d\'eveloppement de syst\`emes informatiques collectifs et coop\'eratifs tr\`es divers. Ainsi, un syst\`eme permettant la r\'eservation de moyens de transports, de chambres d'h\^otel et de services analogues \`a partir d'un terminal unique constitue un outil dont les avantages sont multiples : meilleur emploi des ressources, co\^uts administratifs r\'eduits, commodit\'e pour l'usager... Dans un autre domaine, la simplification des \'echanges d'informations entre les personnes physiques et morales et les administrations (S\'ecurit\'e sociale, Finances, P.T.T., etc) sera assur\'ee par la cr\'eation d'une centrale automatis\'ee d'information assurant le recueil de celle-ci et sa communication \`a chaque demandeur sous la r\'eserve d'un contr\^ole et d'une r\'eglementation assurant un usage conforme aux droits et libert\'es de chacun.

    Ces aper\c{c}us mettent en \'evidence que, pratiquement, toute activit\'e, qu'elle soit \'economique, sociale ou intellectuelle, peut donner lieu \`a des applications de l'ordinateur. Celui-ci se verra confier, outre les t\^aches purement administratives, les cas o\`u les d\'ecisions sont pr\'evisibles et analysables. D'autre part, c'est un moyen d'information consid\'erable puisqu'il peut apporter \`a chacun la documentation souhait\'ee au moment opportun et sans attente. Enfin, comme prolongement de la m\'emoire et de la logique, il constitue pour l'homme une aide dans ses activit\'es les plus cr\'eatrices, qu'elles soient enseignement, recherche... Aujourd'hui, on peut affirmer que les m\'ethodes de travail de chacun seront modifi\'ees du fait de l'existence de ce nouvel outil ; selon les cas, leurs \'evolutions seront profondes ou secondaires, rapides ou progressives, mais elles seront partout pr\'esentes.
        
    \subsection{Titre de sous-section}
    
    Sous-section.
    
    \subsection{Titre de sous-section}

    Sous-section.
        
\section{Titre de section}

    Section.
        
    \subsection{Titre de sous-section}
    
    Sous-section.
    
    \subsection{Titre de sous-section}
    
    Sous-section.
    
    \subsubsection{Titre de sous-sous-section}
    
    Sous-sous-section.
    
    \paragraph{Un paragraphe:} exemple
    
    \subparagraph{Un sous paragraphe:} exemple
    
    \subsubsection{Titre de sous-sous-section}

    Sous-sous-section.
    
    \subsection{Titre de sous-section}
    
    Sous-section.
 % fichier chapitre-1.tex

%=========================== CHAPITRE 2 ============================

\modeFrancais{}
\chapter[Titre court du chapitre 2] 
        {\singlespacing%
         Titre long du chapitre 2}
         \label{ch:chapitre-2}

    Description du chapitre.

\section{Titre de section}

    Section.

\section{Titre de section}
        
    Section.
        
    \subsection{Titre de sous-section}
    
    Sous-section.
    
    \subsection{Titre de sous-section}

    Sous-section.
        
\section{Titre de section}

    Section.
        
    \subsection{Titre de sous-section}
    
    Sous-section.
    
    \subsection{Titre de sous-section}
    
    Sous-section.
    
    \subsubsection{Titre de sous-sous-section}
    
    Sous-sous-section.
    
    \paragraph{Un paragraphe:} exemple
    
    \subparagraph{Un sous paragraphe:} exemple
    
    \subsubsection{Titre de sous-sous-section}

    Sous-sous-section.
    
    \subsection{Titre de sous-section}
    
    Sous-section.
 % fichier chapitre-2.tex

%=========================== CHAPITRE 3 ============================

\modeFrancais{}
\chapter[Titre court du chapitre 3] 
        {\singlespacing%
         Titre long du chapitre 3}
         \label{ch:chapitre-3}

    Description du chapitre.

\section{Titre de section}

    Section.

\section{Titre de section}
        
    Section.
        
    \subsection{Titre de sous-section}
    
    Sous-section.
    
    \subsection{Titre de sous-section}

    Sous-section.
        
\section{Titre de section}

    Section.
        
    \subsection{Titre de sous-section}
    
    Sous-section.
    
    \subsection{Titre de sous-section}
    
    Sous-section.
    
    \subsubsection{Titre de sous-sous-section}
    
    Sous-sous-section.
    
    \paragraph{Un paragraphe:} exemple
    
    \subparagraph{Un sous paragraphe:} exemple
    
    \subsubsection{Titre de sous-sous-section}

    Sous-sous-section.
    
    \subsection{Titre de sous-section}
    
    Sous-section.
 % fichier chapitre-3.tex

%=========================== CONCLUSION ============================

\modeFrancais{}
% DEBUT DE LA CONCLUSION
\Conclusion
    Ceci est la conclusion. 
    Ceci est la conclusion. 
    Ceci est la conclusion. 
    Ceci est la conclusion. 
    Ceci est la conclusion. 
    Ceci est la conclusion. 
    Ceci est la conclusion. 
    Ceci est la conclusion. 
    Ceci est la conclusion. 
    Ceci est la conclusion. 
    Ceci est la conclusion. 
    Ceci est la conclusion. 
    Ceci est la conclusion. 
    Ceci est la conclusion. 
    Ceci est la conclusion. 
    Ceci est la conclusion. 
    Ceci est la conclusion. 
    Ceci est la conclusion. 
    Ceci est la conclusion. 
    Ceci est la conclusion. 
    Ceci est la conclusion. 
    Ceci est la conclusion. 
    Ceci est la conclusion. 
    Ceci est la conclusion. 
    Ceci est la conclusion. 
    Ceci est la conclusion. 
    Ceci est la conclusion. 
    Ceci est la conclusion. 
    Ceci est la conclusion. 
    Ceci est la conclusion. 
    Ceci est la conclusion. 

% FIN DE LA CONCLUSION

%%=================================================================
%%========================== ANNEXES ==============================
%%=================================================================

\appendix
\renewcommand{\chaptermark}[1]{\markboth{\textsc{\appendixname\ \thechapter.\ #1}}{}}
%% Les annexes peuvent \^etre en fran\c{c}ais ou en anglais
%% Si elles sont en anglais, elles doivent contenir \modeAnglais ou \englishMode
%% juste après la commande \chapter{xxx}

%%=========================== ANNEXE A ============================

\modeFrancais{}
\chapter{Premi\`ere annexe}
\label{annexe:A}

Les annexes peuvent \^etre \'ecrites en fran\c{c}ais ou en anglais.

Voici un exemple de citation: on utilise la commande \verb@cite@ pour citer des articles comme ceci~(\cite{Abr.96-BBook})
ou ceci~(\cite{Hoa.85-CSP}). 

On peut aussi en citer plusieurs \`a la fois (liste de citations~
\cite{Jac.83-JSD,Mil.89-CCS,INRIA.cadp})

Ceci appara\^itra dans la bibliographie du document.

Les entr\'ees de cette bibliographie doivent \^etre mises dans un document \texttt{.bib}. Ce gabarit de m\'emoire et de th\`ese en contient un portant le nom de \texttt{bibliographie.bib}. C'est LaTeX qui s'occupe de classer et d'int\'egrer les bons liens dans votre document, ainsi que de g\'en\'erer la bibliographie correctement.

Il y a plusieurs types de documents pouvant \^etre utilis\'es:
\begin{description}
\item[book] un livre (\cite{Abr.96-BBook,Hoa.85-CSP,Jac.83-JSD,Mil.89-CCS})
\item[manual] un manuel de cours, sensiblement \'equivalent \`a un livre
\item[inbook] un chapitre d'un livre (\cite{Ch.96-Programmer-avec-Scheme})
\item[article] un article de recherche classique publi\'e dans un journal (et non une conf\'erence) (\cite{Bo.84-VVS,FFL.05-SOSYM,FSd.03-eb3})
\item[proceedings] les actes d'un conf\'erence (\cite{LeWe.09-IFM,ArGnMa.03-FME}) 
\item[inproceedings] une parution dans un acte de conf\'erence (\cite{Matra.99-Meteor,FF.07-ICFEM,Pn.79-The-Temporal-Semantics-of-Concurrent-Programs}) \'eventuellement avec une r\'ef\'erence crois\'ee  (crossref) (\cite{LeBu.03-ProB})
\item[conference] identique au pr\'ec\'edent
\item[phdthesis] une th\`ese de doctorat (\cite{FRA.06-thesis})
\item[masterthesis] un m\'emoire de ma\^itrise (\cite{Ri.01-EB3})
\item[techreport] un rapport technique paru dans une institution (universitaire ou autre) et disponible publiquement (\cite{FRA.05-TR9})
\item[incollection] un article d'une collection d'articles parus ailleurs (\cite{BB.89-LOTOS}) 
\item[booklet] un livret ou un document comme une th\`ese d'habilitation \`a diriger la recherche ou un manuel utilisateur (\cite{La.02-CDBD,STE.97-Manuel-B})
\item[unpublished] un document non publi\'es, par exemple pour cause de confidentialit\'e (\cite{BuDo.99-guide-B})
\item[misc] n'importe quel autre document, utile pour un site Internet ou un document publi\'e sur un site personnel (\cite{INRIA.cadp,Gi.08-Logic-Vs.-Intelligence})
\end{description} % fichier annexe-A.tex

%%=========================== ANNEXE B ============================

\modeFrancais{}

\chapter{Deuxi\`eme annexe}
\label{annexe:B}

Si l'option \verb@hypertexte@ est d\'eclar\'ee, on peut obtenir des liens avec
la commande \verb@autoref@. Par exemple:
\begin{center}
Le \autoref{ch:chapitre-1} est plac\'e avant l'\autoref{annexe:B}.
\end{center}
Ceci est \`a comparer\`a
\begin{center}
Le chapitre~\ref{ch:chapitre-1} est plac\'e avant l'annexe~\ref{annexe:B}.
\end{center} % fichier annexe-B.tex

%=================================================================
%=================== BIBLIOGRAPHIE ET INDEX ======================
%=================================================================

% --- Bibliographie
\bibliographystyle{style/bibUdeSPlain}
% le contenu est dans un fichier bibliographie.bib. 
% Il est possible de mettre plusieurs noms de fichiers séparés par une virgule.
\bibliography{bibliographie}  
% --- Bibliographie

\end{document}
