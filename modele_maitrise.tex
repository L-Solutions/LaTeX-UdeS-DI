%!TEX TS-program = pdflatex

\documentclass[caractereUtf, % caract\`ere UTF8 activ\'e
               maitrise,     % mod\`ele pour une ma\^itrise
               hypertexte,   % support pour les liens 
               simpleface    % impression sur simple face
               ]{style/scienceUdeS}

%% Gabarit pour maitrise et pour maitrise GL
%%
%% Version 2019/03/25 v2.1
%%
%% Benoît Fraikin
%% benoit.fraikin(at)usherbrooke(dot)ca

% =========================================== Options principales
% - bibliothequeNationale : pr\'esente le document pour la copie de la 
%   biblioth\`eque nationale. Attention, certaines commandes ne
%   fonctionnent plus.
%
% - Natbib : d\'eclare le package Natbib pour \^etre utilis\'e dans le m\'emoire
%
% - caractereUtf : active les caract\`eres cod\'es en UTF8 au lieu du latin1
% - caractereLatin : active les caract\`eres cod\'es en latin1 au lieu du UTF8
%
% - logopdflatex : utilise le logo de l'universit\'e en pdf, pour ceux 
%   qui utilise pdflatex et non latex.
%
% - pasDeFigure : le document ne contient pas de figure (\begin{figure})
% - pasDeTableau : le document ne contient pas de tableau (\begin{table})
% - pasDeListing : le document ne contient pas de listing (\begin{lstlisting})
%
% - maitrise : pour une ma\^itrise en informatique
% - maitriseGL : pour une ma\^itrise en g\'enie logiciel
%                (ne diff\`ere pas de l'option "maitrise" pour l'instant)
% - these : pour une these de doctorat
%
% - enRedaction : option pour signaler l'\'etat du document
%                 \`a retirer pour la version finale
% =========================================== Options secondaires
% - hypertexte
% - rectoverso
% - simpleface
% - draft
% - final
%=================================================================

\begin{document}
%====================== DEBUT DU DOCUMENT ========================
\modeFrancais
% -------------------------------------------------
% PACKAGE SUPPLEMENTAIRE 
% -------------------------------------------------

% Les packages d\'ej\`a charg\'es sont les suivants :
%--> \usepackage[utf8]{inputenc} ou \usepackage[latin1]{inputenc}
%--> \usepackage[T1]{fontenc}
%--> \usepackage{lmodern}
%--> \usepackage{enumerate}
%--> \usepackage{amsfonts, amsmath, amssymb, stmaryrd, latexsym}
%--> \usepackage{xspace, setspace}
%--> \usepackage{array}
%--> \usepackage[dvipsnames,usenames]{color}
%--> \usepackage[table]{xcolor}
%--> \usepackage{wrapfig, epsfig}
%--> \usepackage[english,frenchb]{babel}
%--> \usepackage{frbib} % bibliography en francais
%--> \usepackage{fancyhdr}
%--> \usepackage{geometry}
%--> \usepackage{ulem} % for sout and underline
%--> \renewcommand{\emph}{\textit}
%--> \usepackage{times,helvet,courier}
%--> \usepackage{listings}

% Vous pouvez ajouter des nouveaux packages ici
% \

% -------------------------------------------------
% COMMANDE SUPPLEMENTAIRE
% -------------------------------------------------

%---- pour l'anglais ----
\newcommand{\ie}{{i.e.}\xspace}
\newcommand{\eg}{{e.g.}\xspace}
\newcommand{\cf}{{\it cf.}\xspace}

%---- pour le francais ----
\newcommand{\cad}{{c'est-\`a-dire}\xspace}
\newcommand{\Cad}{{C'est-\`a-dire}\xspace}
\newcommand{\etc}{{\it etc.}\xspace}

%---- Math ----
\newcommand{\N}{\ensuremath{\mathbb{N}}\xspace}
\newcommand{\Z}{\ensuremath{\mathbb{Z}}\xspace}

% Nouvelles d\'efintion d'environnement de th\'eor\`eme et de d\'efinition.
\newtheorem{frtheoreme}{Th\'eor\`eme}[section]
\newtheorem{frlemme}[frtheoreme]{Lemme}
\newtheorem{frprop}[frtheoreme]{Proposition}
\newtheorem{frcoro}[frtheoreme]{Corollaire}
\newtheorem{frdefinition}{D\'efinition}[section]

% -------------------------------------------------
% PAGE DE TITRE
% -------------------------------------------------

% ATTENTION : ne pas utiliser les macros \title \author ou \date !!!
\auteur{Beno\^it Fraikin}
\titre{Ma ma\^itrise ou ma th\`ese}

% -------------------------------------------------

% DESCRIPTION DU SOMMAIRE (EN FRANCAIS) -----------
\MotsCles{mots-cles1; mots-cles2.}
\sommaire{Ceci est le sommaire. Il est en fran\c{c}ais.}

% REMERCIEMENTS -----------------------------------
\remerciements{Ceci est la liste des remerciements. Ils sont en fran\c{c}ais.}

% LISTE DES ABREVIATIONS --------------------------
\abreviations{ 
    \begin{description}
    \item[SW] Star Wars
    \item[ST] Star Trek
    \item[BV] Babylon V
    \end{description}
}

% !TEX root =  ../modele_these.tex
% DEBUT DE LA PAGE DU JURY

\DecisionDuJury
{%
\today
}%
{%
\Directeur{[Pr\'enom et Nom]}{[Nom]}
%\Directrice{[Pr\'enom et Nom]}{[Nom]}

\Codirectrice{[Pr\'enom et Nom]}{[Nom]}
%\Codirecteur{[Pr\'enom et Nom]}{[Nom]}

\MembreInterne{[Pr\'enom et Nom]}{[Nom]}

\MembreExterne{[Pr\'enom et Nom]}{[Titre]}{[Nom]}{[provenance]}

\President{[Pr\'enom et Nom]}{[Nom]}
}%
% FIN DE LA PAGE DU JURY


%=================================================================
\modeFrancais{}
\enteteDeLaThese{}
\renewcommand{\chaptermark}[1]{\markboth{\textsc{\chaptername\ \thechapter.\ #1}}{}}
\renewcommand{\sectionmark}[1]{\markright{\textsc{\thesection. #1}}	}
    
%========================== INTRODUCTION ===========================

\modeFrancais{}
% DEBUT DE L'INTRODUCTION
\Introduction
    Ceci est l'introduction.
    Ceci est l'introduction.
    Ceci est l'introduction.
    Ceci est l'introduction.
    Ceci est l'introduction.
    Ceci est l'introduction.
    Ceci est l'introduction.
    Ceci est l'introduction.
    Ceci est l'introduction.
    Ceci est l'introduction.
    Ceci est l'introduction.
    Ceci est l'introduction.
    Ceci est l'introduction.
    Ceci est l'introduction.
    Ceci est l'introduction.
    Ceci est l'introduction.
    Ceci est l'introduction.
    Ceci est l'introduction.
    Ceci est l'introduction.
    Ceci est l'introduction.
    Ceci est l'introduction.
    Ceci est l'introduction.
    Ceci est l'introduction.
    Ceci est l'introduction.
    Ceci est l'introduction.
    Ceci est l'introduction.
    Ceci est l'introduction.
    Ceci est l'introduction.
    Ceci est l'introduction.
    Ceci est l'introduction.
    Ceci est l'introduction.
    Ceci est l'introduction.
    Ceci est l'introduction.
    Ceci est l'introduction.
    Ceci est l'introduction.
    Ceci est l'introduction.
    Ceci est l'introduction.
    Ceci est l'introduction.
    Ceci est l'introduction.
    Ceci est l'introduction.
    Ceci est l'introduction.
    Ceci est l'introduction.
    Ceci est l'introduction.
    Ceci est l'introduction.
    Ceci est l'introduction.
    Ceci est l'introduction.
    Ceci est l'introduction.
    Ceci est l'introduction.
    Ceci est l'introduction.
    Ceci est l'introduction.
    Ceci est l'introduction.
    Ceci est l'introduction.
    Ceci est l'introduction.
    Ceci est l'introduction.
    Ceci est l'introduction.
    Ceci est l'introduction.
    Ceci est l'introduction.
    Ceci est l'introduction.
    Ceci est l'introduction.
    Ceci est l'introduction.
    Ceci est l'introduction.
    Ceci est l'introduction.
    Ceci est l'introduction.
    Ceci est l'introduction.
    Ceci est l'introduction.
    Ceci est l'introduction.
    Ceci est l'introduction.
    Ceci est l'introduction.
    Ceci est l'introduction.
    Ceci est l'introduction.
    Ceci est l'introduction.
    Ceci est l'introduction.
    Ceci est l'introduction.
    Ceci est l'introduction.
    Ceci est l'introduction.
    Ceci est l'introduction.
    Ceci est l'introduction.
    Ceci est l'introduction.
    Ceci est l'introduction.
    Ceci est l'introduction.
    Ceci est l'introduction.
    Ceci est l'introduction.
    Ceci est l'introduction.
    Ceci est l'introduction.
    Ceci est l'introduction.
    Ceci est l'introduction.
    Ceci est l'introduction.
    Ceci est l'introduction.
    Ceci est l'introduction.
    Ceci est l'introduction.
    Ceci est l'introduction.
    Ceci est l'introduction.
    Ceci est l'introduction.
    Ceci est l'introduction.
    Ceci est l'introduction.
    Ceci est l'introduction.
    Ceci est l'introduction.
    Ceci est l'introduction.
    Ceci est l'introduction.
    Ceci est l'introduction.
    Ceci est l'introduction.
    Ceci est l'introduction.
    Ceci est l'introduction.
    Ceci est l'introduction.
    Ceci est l'introduction.
    Ceci est l'introduction.
    Ceci est l'introduction.
    Ceci est l'introduction.
    Ceci est l'introduction.
    Ceci est l'introduction.
    Ceci est l'introduction.
    Ceci est l'introduction.
    Ceci est l'introduction.
    Ceci est l'introduction.
    Ceci est l'introduction.
    Ceci est l'introduction.
    Ceci est l'introduction.
    Ceci est l'introduction.
    Ceci est l'introduction.
    Ceci est l'introduction.
    Ceci est l'introduction.
    Ceci est l'introduction.
    Ceci est l'introduction.
    Ceci est l'introduction.
    Ceci est l'introduction.
    Ceci est l'introduction.
    Ceci est l'introduction.
    Ceci est l'introduction.
    Ceci est l'introduction.
    Ceci est l'introduction.
    Ceci est l'introduction.
    Ceci est l'introduction.
    Ceci est l'introduction.
    Ceci est l'introduction.
    Ceci est l'introduction.
    Ceci est l'introduction.
    Ceci est l'introduction.
    Ceci est l'introduction.
    Ceci est l'introduction.
    Ceci est l'introduction.
    Ceci est l'introduction.
    Ceci est l'introduction.
    Ceci est l'introduction.
    Ceci est l'introduction.
    Ceci est l'introduction.
    Ceci est l'introduction.
    Ceci est l'introduction.
    Ceci est l'introduction.
    Ceci est l'introduction.
    Ceci est l'introduction.
% FIN DE L'INTRODUCTION
   
% Un article par chapitre
%=========================== CHAPITRE 1 ============================

\modeFrancais{}
\chapter[Titre court de l'article 1] 
        {\singlespacing%
         Titre long en francais de l'article 1}
         \label{ch:chapitre-1}
        
\resumeArticle{R\'esum\'e de l'article}
               
\commentairesArticle{L'article a-t-il \'et\'e publi\'e ou soumis dans une revue ou une conf\'erence?
                     
                     Quelle est la part de l'\'etudiant ou de l'\'etudiante?
                  
                     Quel est l'apport de l'article par rapport à la th\`ese?
                    }

\cleardoublepage 


% - choisir la bonne option parmi les deux suivantes
%\modeAnglais
\modeFrancais
% --------------------------------------------

% IMPORTANT : ARTICLE TEL QUE SOUMIS (ET NON TEL QUE PUBLI\'E)
\titleArticle{Titre de l'article 1 soumis}

\authorArticle{
  {\large Auteur}\\
  D\'{e}partement d'informatique, 
  Universit\'{e} de Sherbrooke,\\
  Sher\-brooke, Qu\'{e}\-bec, Canada J1K~2R1\\
  \email{xxx.yyy@usherbrooke.ca}\\[1ex]
  
  {\large co-auteur(e)}\\
  Adresse\\
  \email{courriel}
}

\keywordsArticle{ keywords }

\begin{abstractArticle}
Abstract
\end{abstractArticle} % fichier chapitre-1.tex

%=========================== CHAPITRE 2 ============================

\modeFrancais{}
\chapter[Titre court du chapitre 2] 
        {\singlespacing%
         Titre long du chapitre 2}
         \label{ch:chapitre-2}

    Description du chapitre.

\section{Titre de section}

    Section.

\section{Titre de section}
        
    Section.
        
    \subsection{Titre de sous-section}
    
    Sous-section.
    
    \subsection{Titre de sous-section}

    Sous-section.
        
\section{Titre de section}

    Section.
        
    \subsection{Titre de sous-section}
    
    Sous-section.
    
    \subsection{Titre de sous-section}
    
    Sous-section.
    
    \subsubsection{Titre de sous-sous-section}
    
    Sous-sous-section.
    
    \paragraph{Un paragraphe:} exemple
    
    \subparagraph{Un sous paragraphe:} exemple
    
    \subsubsection{Titre de sous-sous-section}

    Sous-sous-section.
    
    \subsection{Titre de sous-section}
    
    Sous-section.
 % fichier chapitre-2.tex

%=========================== CHAPITRE 3 ============================

\modeFrancais{}
\include{contenu_maitrise/chapitre-3} % fichier chapitre-3.tex

%=========================== CONCLUSION ============================

\modeFrancais{}
% DEBUT DE LA CONCLUSION
\Conclusion
    Ceci est la conclusion. 
    Ceci est la conclusion. 
    Ceci est la conclusion. 
    Ceci est la conclusion. 
    Ceci est la conclusion. 
    Ceci est la conclusion. 
    Ceci est la conclusion. 
    Ceci est la conclusion. 
    Ceci est la conclusion. 
    Ceci est la conclusion. 
    Ceci est la conclusion. 
    Ceci est la conclusion. 
    Ceci est la conclusion. 
    Ceci est la conclusion. 
    Ceci est la conclusion. 
    Ceci est la conclusion. 
    Ceci est la conclusion. 
    Ceci est la conclusion. 
    Ceci est la conclusion. 
    Ceci est la conclusion. 
    Ceci est la conclusion. 
    Ceci est la conclusion. 
    Ceci est la conclusion. 
    Ceci est la conclusion. 
    Ceci est la conclusion. 
    Ceci est la conclusion. 
    Ceci est la conclusion. 
    Ceci est la conclusion. 
    Ceci est la conclusion. 
    Ceci est la conclusion. 
    Ceci est la conclusion. 

% FIN DE LA CONCLUSION

%%=================================================================
%%========================== ANNEXES ==============================
%%=================================================================

\appendix
\renewcommand{\chaptermark}[1]{\markboth{\textsc{\appendixname\ \thechapter.\ #1}}{}}
%% Les annexes peuvent \^etre en fran\c{c}ais ou en anglais
%% Si elles sont en anglais, elles doivent contenir \modeAnglais ou \englishMode
%% juste après la commande \chapter{xxx}

%%=========================== ANNEXE A ============================

\modeFrancais{}
\chapter{Premi\`ere annexe}
\label{annexe:A}

Les annexes peuvent \^etre \'ecrites en fran\c{c}ais ou en anglais.

Voici un exemple de citation: on utilise la commande \verb@cite@ pour citer des articles comme ceci~(\cite{Abr.96-BBook})
ou ceci~(\cite{Hoa.85-CSP}). 

On peut aussi en citer plusieurs \`a la fois (liste de citations~
\cite{Jac.83-JSD,Mil.89-CCS,INRIA.cadp})

Ceci appara\^itra dans la bibliographie du document.

Les entr\'ees de cette bibliographie doivent \^etre mises dans un document \texttt{.bib}. Ce gabarit de m\'emoire et de th\`ese en contient un portant le nom de \texttt{bibliographie.bib}. C'est LaTeX qui s'occupe de classer et d'int\'egrer les bons liens dans votre document, ainsi que de g\'en\'erer la bibliographie correctement.

Il y a plusieurs types de documents pouvant \^etre utilis\'es:
\begin{description}
\item[book] un livre (\cite{Abr.96-BBook,Hoa.85-CSP,Jac.83-JSD,Mil.89-CCS})
\item[manual] un manuel de cours, sensiblement \'equivalent \`a un livre
\item[inbook] un chapitre d'un livre (\cite{Ch.96-Programmer-avec-Scheme})
\item[article] un article de recherche classique publi\'e dans un journal (et non une conf\'erence) (\cite{Bo.84-VVS,FFL.05-SOSYM,FSd.03-eb3})
\item[proceedings] les actes d'un conf\'erence (\cite{LeWe.09-IFM,ArGnMa.03-FME}) 
\item[inproceedings] une parution dans un acte de conf\'erence (\cite{Matra.99-Meteor,FF.07-ICFEM,Pn.79-The-Temporal-Semantics-of-Concurrent-Programs}) \'eventuellement avec une r\'ef\'erence crois\'ee  (crossref) (\cite{LeBu.03-ProB})
\item[conference] identique au pr\'ec\'edent
\item[phdthesis] une th\`ese de doctorat (\cite{FRA.06-thesis})
\item[masterthesis] un m\'emoire de ma\^itrise (\cite{Ri.01-EB3})
\item[techreport] un rapport technique paru dans une institution (universitaire ou autre) et disponible publiquement (\cite{FRA.05-TR9})
\item[incollection] un article d'une collection d'articles parus ailleurs (\cite{BB.89-LOTOS}) 
\item[booklet] un livret ou un document comme une th\`ese d'habilitation \`a diriger la recherche ou un manuel utilisateur (\cite{La.02-CDBD,STE.97-Manuel-B})
\item[unpublished] un document non publi\'es, par exemple pour cause de confidentialit\'e (\cite{BuDo.99-guide-B})
\item[misc] n'importe quel autre document, utile pour un site Internet ou un document publi\'e sur un site personnel (\cite{INRIA.cadp,Gi.08-Logic-Vs.-Intelligence})
\end{description} % fichier annexe-A.tex

%%=========================== ANNEXE B ============================

\modeFrancais{}

\chapter{Deuxi\`eme annexe}
\label{annexe:B}

Si l'option \verb@hypertexte@ est d\'eclar\'ee, on peut obtenir des liens avec
la commande \verb@autoref@. Par exemple:
\begin{center}
Le \autoref{ch:chapitre-1} est plac\'e avant l'\autoref{annexe:B}.
\end{center}
Ceci est à comparer à
\begin{center}
Le chapitre~\ref{ch:chapitre-1} est plac\'e avant l'annexe~\ref{annexe:B}.
\end{center} % fichier annexe-B.tex

%=================================================================
%=================== BIBLIOGRAPHIE ET INDEX ======================
%=================================================================

% --- Bibliographie
\bibliographystyle{style/bibUdeSPlain}
% le contenu est dans un fichier bibliographie.bib. 
% Il est possible de mettre plusieurs noms de fichiers séparés par une virgule.
\bibliography{bibliographie}  
% --- Bibliographie

\end{document}
