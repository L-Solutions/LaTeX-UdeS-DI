% -------------------------------------------------
% PACKAGE SUPPLEMENTAIRE 
% -------------------------------------------------

% Les packages d\'ej\`a charg\'es sont les suivants :
%--> \usepackage[utf8]{inputenc} ou \usepackage[latin1]{inputenc}
%--> \usepackage[T1]{fontenc}
%--> \usepackage{lmodern}
%--> \usepackage{enumerate}
%--> \usepackage{amsfonts, amsmath, amssymb, stmaryrd, latexsym}
%--> \usepackage{xspace, setspace}
%--> \usepackage{array}
%--> \usepackage[dvipsnames,usenames]{color}
%--> \usepackage[table]{xcolor}
%--> \usepackage{wrapfig, epsfig}
%--> \usepackage[english,frenchb]{babel}
%--> \usepackage{frbib} % bibliography en francais
%--> \usepackage{fancyhdr}
%--> \usepackage{geometry}
%--> \usepackage{ulem} % for sout and underline
%--> \renewcommand{\emph}{\textit}
%--> \usepackage{times,helvet,courier}
%--> \usepackage{listings}

% Vous pouvez ajouter des nouveaux packages ici
% \

% -------------------------------------------------
% COMMANDE SUPPLEMENTAIRE
% -------------------------------------------------

%---- pour l'anglais ----
\newcommand{\ie}{{i.e.}\xspace}
\newcommand{\eg}{{e.g.}\xspace}
\newcommand{\cf}{{\it cf.}\xspace}

%---- pour le francais ----
\newcommand{\cad}{{c'est-\`a-dire}\xspace}
\newcommand{\Cad}{{C'est-\`a-dire}\xspace}
\newcommand{\etc}{{\it etc.}\xspace}

%---- Math ----
\newcommand{\N}{\ensuremath{\mathbb{N}}\xspace}
\newcommand{\Z}{\ensuremath{\mathbb{Z}}\xspace}

% Nouvelles d\'efintion d'environnement de th\'eor\`eme et de d\'efinition.
\newtheorem{frtheoreme}{Th\'eor\`eme}[section]
\newtheorem{frlemme}[frtheoreme]{Lemme}
\newtheorem{frprop}[frtheoreme]{Proposition}
\newtheorem{frcoro}[frtheoreme]{Corollaire}
\newtheorem{frdefinition}{D\'efinition}[section]

% -------------------------------------------------
% PAGE DE TITRE
% -------------------------------------------------

% ATTENTION : ne pas utiliser les macros \title \author ou \date !!!
\auteur{Beno\^it Fraikin}
\titre{Ma ma\^itrise ou ma th\`ese}

% -------------------------------------------------

% DESCRIPTION DU SOMMAIRE (EN FRANCAIS) -----------
\MotsCles{mots-cles1; mots-cles2.}
\sommaire{Ceci est le sommaire. Il est en fran\c{c}ais.}

% REMERCIEMENTS -----------------------------------
\remerciements{Ceci est la liste des remerciements. Ils sont en fran\c{c}ais.}

% LISTE DES ABREVIATIONS --------------------------
\abreviations{ 
    \begin{description}
    \item[SW] Star Wars
    \item[ST] Star Trek
    \item[BV] Babylon V
    \end{description}
}